% Options for packages loaded elsewhere
\PassOptionsToPackage{unicode}{hyperref}
\PassOptionsToPackage{hyphens}{url}
%
\documentclass[
]{book}
\usepackage{lmodern}
\usepackage{amssymb,amsmath}
\usepackage{ifxetex,ifluatex}
\ifnum 0\ifxetex 1\fi\ifluatex 1\fi=0 % if pdftex
  \usepackage[T1]{fontenc}
  \usepackage[utf8]{inputenc}
  \usepackage{textcomp} % provide euro and other symbols
\else % if luatex or xetex
  \usepackage{unicode-math}
  \defaultfontfeatures{Scale=MatchLowercase}
  \defaultfontfeatures[\rmfamily]{Ligatures=TeX,Scale=1}
\fi
% Use upquote if available, for straight quotes in verbatim environments
\IfFileExists{upquote.sty}{\usepackage{upquote}}{}
\IfFileExists{microtype.sty}{% use microtype if available
  \usepackage[]{microtype}
  \UseMicrotypeSet[protrusion]{basicmath} % disable protrusion for tt fonts
}{}
\makeatletter
\@ifundefined{KOMAClassName}{% if non-KOMA class
  \IfFileExists{parskip.sty}{%
    \usepackage{parskip}
  }{% else
    \setlength{\parindent}{0pt}
    \setlength{\parskip}{6pt plus 2pt minus 1pt}}
}{% if KOMA class
  \KOMAoptions{parskip=half}}
\makeatother
\usepackage{xcolor}
\IfFileExists{xurl.sty}{\usepackage{xurl}}{} % add URL line breaks if available
\IfFileExists{bookmark.sty}{\usepackage{bookmark}}{\usepackage{hyperref}}
\hypersetup{
  pdftitle={Aide-mémoire : Théories et pratiques sur R},
  pdfauthor={Alexis Mérot},
  hidelinks,
  pdfcreator={LaTeX via pandoc}}
\urlstyle{same} % disable monospaced font for URLs
\usepackage{longtable,booktabs}
% Correct order of tables after \paragraph or \subparagraph
\usepackage{etoolbox}
\makeatletter
\patchcmd\longtable{\par}{\if@noskipsec\mbox{}\fi\par}{}{}
\makeatother
% Allow footnotes in longtable head/foot
\IfFileExists{footnotehyper.sty}{\usepackage{footnotehyper}}{\usepackage{footnote}}
\makesavenoteenv{longtable}
\usepackage{graphicx}
\makeatletter
\def\maxwidth{\ifdim\Gin@nat@width>\linewidth\linewidth\else\Gin@nat@width\fi}
\def\maxheight{\ifdim\Gin@nat@height>\textheight\textheight\else\Gin@nat@height\fi}
\makeatother
% Scale images if necessary, so that they will not overflow the page
% margins by default, and it is still possible to overwrite the defaults
% using explicit options in \includegraphics[width, height, ...]{}
\setkeys{Gin}{width=\maxwidth,height=\maxheight,keepaspectratio}
% Set default figure placement to htbp
\makeatletter
\def\fps@figure{htbp}
\makeatother
\setlength{\emergencystretch}{3em} % prevent overfull lines
\providecommand{\tightlist}{%
  \setlength{\itemsep}{0pt}\setlength{\parskip}{0pt}}
\setcounter{secnumdepth}{5}
\usepackage{booktabs}
\usepackage[french]{babel} % Ajout du français
\usepackage[style=authoryear,]{biblatex}
\addbibresource{book.bib}
\addbibresource{packages.bib}

\title{Aide-mémoire : Théories et pratiques sur R}
\author{Alexis Mérot}
\date{Modifié le : 2020-07-30}

\begin{document}
\maketitle

{
\setcounter{tocdepth}{1}
\tableofcontents
}
\hypertarget{introduction}{%
\chapter*{Introduction}\label{introduction}}
\addcontentsline{toc}{chapter}{Introduction}

Ce projet est un ensemble de notes écrites en R Markdown \autocite{R-rmarkdown} et via
le package \textbf{bookdown} (\url{https://github.com/rstudio/bookdown}). Ces notes
s'accumuleront au fur et à mesure de mon apprentissage des différents outils et
concepts dont j'ai besoin pour les analyses de données et la programmation. Cela
me permet de les comprendre, les mémoriser, ainsi que de les partager.

Le projet s'insérera peut-être dans un autre plus gros projet : la création d'un
blog répertoriant tous mes projets et mon CV. Il commencera certainement lorsque
je démarrerai la lecture de la documentation de l'excellent package \textbf{blogdown}
(\url{https://bookdown.org/yihui/blogdown/}).

\hypertarget{rmarkdown}{%
\chapter{R Markdown, Bookdown \& Blogdown}\label{rmarkdown}}

\hypertarget{pourquoi-rmarkdown}{%
\section{Pourquoi R Markdown ?}\label{pourquoi-rmarkdown}}

R Markdown est un format de fichier (à l'extension \texttt{.Rmd}) fournissant un cadre
de création pour faire des rapports scientifiques automatisés. Ces documents
peuvent ainsi être totalement reproductibles et plusieurs formats de rendu
finale (statiques ou dynamiques) sont supportés.

Le fichier est écrit via le langage Markdown et des sections de code R peuvent y
être insérées facilement (ainsi que du code écrit via d'autres langages tels
que Python ou SQL). Cela offre une syntaxe facile à lire et à écrire tout en
permettant de générer un rapport structuré et élégant.

Pour que cela fonctionne, R Markdown est lié à deux packages : \texttt{knitr} et le
convertisseur universel de document \texttt{pandoc}.\\
Le package \texttt{knitr} permet la création, à partir du fichier \texttt{.Rmd}, d'un fichier
au format \texttt{md} contenant le code et sa sortie. Ce fichier est alors converti
dans le format de rendu final voulu via \texttt{pandoc} (\texttt{.html}, \texttt{.pdf}, etc).

\begin{figure}
\centering
\includegraphics{image/rmarkdownflow.png}
\caption{\emph{Source : \url{https://rmarkdown.rstudio.com/lesson-2.html}}}
\end{figure}

Toutes mes notes seront donc écrites via R Markdown, et cette section intégrera
toutes les astuces intéressantes que je rencontre au fur et à mesure des
besoins.

Pour ne pas paraphraser tout le livre de Yihui Xie, je vous invite à
lire son excellent guide gratuit : \url{https://bookdown.org/yihui/rmarkdown/}.

\hypertarget{statistique}{%
\chapter{Statistique}\label{statistique}}

\hypertarget{SIG}{%
\chapter{Système d'Information Géographique}\label{SIG}}

\hypertarget{web-sources}{%
\chapter{Ressources Internet}\label{web-sources}}

\printbibliography

\end{document}
